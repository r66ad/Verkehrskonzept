\section{Verkehrskonzept Kövenig 2025}	

Seit Jahren wird die Problematik der Raserei in Kövenig ignoriert – mit fatalen Konsequenzen für die Verkehrssicherheit. Weder gibt es regelmäßige Geschwindigkeitskontrollen, noch wurden bauliche Maßnahmen zur Verkehrsberuhigung umgesetzt. Besonders gravierend ist die fehlende Rücksicht auf schwächere Verkehrsteilnehmer: Radfahrer und Fußgänger werden auf den engen Straßen regelmäßig bedrängt und riskant überholt. Das derzeitige (nicht-vorhandene) Verkehrskonzept setzt weiterhin auf eine autozentrierte Planung und ignoriert die Notwendigkeit sicherer, nachhaltiger Mobilität. \footnote{\href{https://katja-diehl.de/}{Katja Diehl}}
\begin{wrapfigure}{r}{7cm}
\includegraphics[width=7cm]{"bild7"}
\end{wrapfigure}
\textbf{Ein Paradebeispiel für diese Fehlplanung ist die nicht realisierte Fahrradstraße zwischen Reil und Kövenig.} Hier fehlte es nicht nur an politischem Willen, sondern auch an klarer Kommunikation und Bürgerbeteiligung. Wenn wir es ernst meinen mit Verkehrssicherheit und Klimaschutz, dann müssen wir endlich den öffentlichen Raum zugunsten aller Verkehrsteilnehmer umgestalten. Dabei muss die Verkehrswende kein städtisches Phänomen bleiben – sie beginnt genau hier, in den ländlichen Räumen, wo viele Wege noch immer ganz selbstverständlich mit dem Auto zurückgelegt werden.

Ein besonders eindrückliches Beispiel für den dringenden Handlungsbedarf ist die Stelle an der Treppe zur Fähre: Familien mit Kindern stehen beim Verlassen der Fähre plötzlich mitten auf der Straße – ohne jede bauliche Sicherung. Es kam bereits mehrfach zu Beinahe-Unfällen, da Autofahrer dort keinerlei Rücksicht nehmen. Es ist allein dem Zufall zu verdanken, dass hier noch nichts Schlimmeres passiert ist. Auch dieser Bereich muss dringend umgestaltet und gesichert werden, um die Sicherheit aller zu gewährleisten (siehe Abbildung ~\ref{fig:Reil-Koevenig}) Plateauaufpflasterung Rechts

\section{Verkrehrsplanung Kövenig}
Seit meinem Umzug nach Kövenig vor vier Jahren verfolge ich das tägliche Verkehrsgeschehen mit wachsender Sorge – und \textbf{wachsendem Frust}. Denn obwohl ich immer wieder konkrete Hinweise und Verbesserungsvorschläge an die zuständigen Behörden und die kommunale Politik weitergeleitet habe, blieb jegliche Rückmeldung aus. Was mich besonders irritiert: Es fehlt nicht an Informationen, sondern am politischen Willen, sie ernst zu nehmen. Die Verkehrswende scheitert in Kövenig nicht an der Komplexität der Aufgabe, sondern am Beharrungsvermögen der Entscheidenden – und an der \textbf{Bequemlichkeit einer autozentrierten Normalität}.

Das Problem ist kein abstraktes. In meinem Konzept \footnote{\href{https://github.com/r66ad/Verkehrskonzept}{Verkehrskonzept}} von 2023 dokumentierte ich zahlreiche Geschwindigkeitsüberschreitungen auf der Strecke zwischen Reil und Kövenig – darunter die meisten Fahrzeuge mit ZEL-, WIL- und BKS-Kennzeichen. Es sind also keineswegs "Auswärtige", die hier zur Gefahr werden. Es sind Menschen aus den umliegenden Dörfern, die sich an eine Infrastruktur gewöhnt haben, die Rücksichtslosigkeit nicht nur duldet, sondern begünstigt.

Die Straße zwischen Reil und Kövenig steht nun vor einer Sanierung – vermutlich mit Verbreiterung, was zur weiteren Flächenversiegelung führt. Diese Maßnahme würde nicht entschleunigen, sondern im Gegenteil den Verkehr noch mehr beschleunigen – mit all den negativen Konsequenzen für Anwohner, Radfahrende und Erholungssuchende. Ich selbst bin dort häufig zu Fuß oder mit dem Fahrrad unterwegs – es ist erschreckend, wenn man auf einer 70er-Strecke mit 100 km/h überholt wird – oft mit weniger als einem Meter Abstand. Familien mit Kindern sollen hier Urlaub machen? Eine solche Erfahrung, und sie kommen kein zweites Mal.

Das häufig vorgebrachte Argument, dass Radfahrende doch die andere Moselseite nutzen könnten, ist realitätsfern. Die Fähre fährt nicht durchgehend und es gibt keine feste Brückenverbindung. Zudem ist die Strecke zwischen Enkirch und Trarbach für Radfahrer noch gefährlicher. Gerade die Route zwischen Reil und Kövenig wäre eigentlich angenehm zu fahren – wenn der motorisierte Verkehr nicht so dominant wäre.

Nach der Strecke folgt der Ortseingang Kövenig – dort wird weiter gerast. Zwei unübersichtliche Zufahrten erhöhen das Risiko zusätzlich. Mein Vorschlag: Tempo 30 für den gesamten Ort – wie es eigentlich für alle Moselorte gelten sollte. Ergänzend sollte ein Plateau den Verkehr am Ortseingang aus Reil abbremsen. Auch der Parkplatz an der Mosel (siehe Abbildung~\ref{fig:Parkplatz}) kann zur Verkehrsberuhigung beitragen: Parkende Fahrzeuge verengen dort gezielt die Fahrbahn. Leider wurde bei der Einrichtung des Parkplatzes ein wertvoller Grünstreifen zerstört – eine Maßnahme, die im Widerspruch zum 2019 ausgerufenen Klimanotstand steht. Ich schlage vor, Teile des Parkplatzes zu renaturieren und künftig nur noch kleine Fahrzeuge (bis 4,5m Länge) in Schrägaufstellung zuzulassen. Größere Fahrzeuge sollten beim Gemeindehaus parken.

Ein weiteres Plateau muss auf Höhe der Treppe zur Fähre errichtet werden: Hier verlassen täglich Familien mit Kindern die Fähre – und stehen ohne jede Sicherung direkt auf der Straße. Es ist ein Wunder, dass es noch keinen schweren Unfall gab. Weitere Plateaus sollten in regelmäßigen Abständen auf der gesamten K65 durch Kövenig folgen.

Auch für die Strecke zwischen Kövenig und der Schleuse Enkirch gilt: Tempo 30 und Überholverbot für Radfahrer (zumindest zwischen Ostern und Oktober). An der Abbiegung zur Schleuse (siehe Abbildung~\ref{fig:Schleuse-Koevenig} - gelbe Markierung) sind weitere bauliche Maßnahmen erforderlich, um Radfahrende aktiv zu schützen.

Diese Empfehlungen  sind nicht nur auf Kövenig beschränkt, sondern könnten – mit Anpassungen – auch auf andere Moselorte übertragen werden. Es ist höchste Zeit, dass Politik und Verwaltung handeln – für mehr Sicherheit, mehr Lebensqualität und ein klares Bekenntnis zur Vision Zero: Keine Toten im Straßenverkehr. \footnote{\href{https://www.quarks.de/podcast/vision-zero-ist-verkehr-ohne-tote-machbar-quarks-daily-spezial/}{Quarks Daily Spezial: Vision Zero}}


	\begin{figure}[H]
		\centering
		\includegraphics[width=0.8\linewidth]{"bild6"}
		\caption[Strecke zwischen Reil und Kövenig]{\textbf{Strecke zwischen Reil und Kövenig} - Diese Strecke braucht dringend eine Geschwindigkeitsreduzierung auf 30 km/h sowie ein absolutes Überholverbot von Radfahrern zwischen Ostern und Oktober. Nur so können wir verhindern, dass Radfahrer weiterhin gefährlich bedrängt werden. Die Umsetzung muss jetzt erfolgen – nicht erst, wenn ein schwerer Unfall passiert.}
		\label{fig:Reil-Koevenig}
	\end{figure}
	
	\begin{figure}[H]
		\centering
		\includegraphics[width=0.8\linewidth]{"bild1"}
		\caption[Parkplatz an der Mosel in Kövenig]{\textbf{Parkplatz an der Mosel in Kövenig} - Statt den Autoverkehr weiterhin zu priorisieren, sollte der Parkplatz so umgestaltet werden, dass er aktiv zur Verkehrsberuhigung beiträgt. Eine smarte Lösung wäre es, die Fahrbahn durch parkende Autos gezielt zu verengen. So wird automatisch ein langsameres und vorsichtigeres Fahren erzwungen, was die Sicherheit für Fußgänger und Radfahrer deutlich erhöht.}
		\label{fig:Parkplatz}
	\end{figure}

	\begin{figure}[H]
		\centering
		\includegraphics[width=0.8\linewidth]{"bild2"}
		\caption[Engstelle in Kövenig]{\textbf{Engstelle in Kövenig} -	Verkehrsberuhigende Elemente wie Plateaus müssen konsequent eingesetzt werden, um den Durchgangsverkehr auszubremsen. Die Erfahrung aus anderen Kommunen zeigt, dass bauliche Maßnahmen die wirksamste Methode sind, um Geschwindigkeitsbegrenzungen tatsächlich durchzusetzen. Besonders an der Ortseinfahrt aus Reil sowie in regelmäßigen Abständen innerhalb des Ortskerns müssen solche Maßnahmen umgesetzt werden. }
		\label{fig:Engstelle-Koevenig}
	\end{figure}
	
	\begin{figure}[H]
		\centering
		\includegraphics[width=0.8\linewidth]{"bild3"}
		\caption[Kövenig Ortsausfahrt Richtung Traben]{\textbf{Kövenig Ortsausfahrt Richtung Traben} - Auch hier muss eine klare Priorisierung der schwächeren Verkehrsteilnehmer erfolgen. Ein Umbau der Straßenführung mit mehr Querungshilfen und optischen Verengungen wäre ein erster sinnvoller Schritt.}
		\label{fig:Koevenig-TT}
	\end{figure}
	
	\begin{figure}[H]
		\centering
		\includegraphics[width=0.8\linewidth]{"bild4"}
		\caption[Strecke zwischen Kövenig und Schleuse Enkirch]{\textbf{Strecke zwischen Kövenig und der Schleuse Enkirch} - 	Die derzeitige Situation ist nicht tragbar: Hohe Geschwindigkeiten, riskante Überholmanöver und eine Infrastruktur, die Radfahrer in gefährliche Situationen bringt. Eine Temporeduzierung auf 30 km/h und ein Überholverbot für Radfahrer während der Hauptverkehrszeit zwischen Ostern und Oktober sind unverzichtbare Maßnahmen. Nur so kann eine sichere und gerechte Nutzung dieser Strecke für alle Verkehrsteilnehmer gewährleistet werden.}
		\label{fig:Schleuse-Koevenig}
	\end{figure}

	\begin{figure}[H]
		\centering
		\includegraphics[width=0.8\linewidth]{"bild5"}
		\caption[Fahrradstrecke weiter Richtung Traben]{\textbf{Fahrradstrecke weiter Richtung Traben} -  Verkehrsführung der Radfahrer hinter Kövenig in Richtung Traben: Die gelb markierte Route verläuft direkt auf die Landstraße. Hier rasen viele Fahrzeuge mit hoher Geschwindigkeit vorbei – eine gefährliche Situation insbesondere für Familien mit Kindern.}
		\label{fig:Koevenig-Traben}
	\end{figure}
	\section{Schlussfolgerung}
	Die Politik vor Ort muss endlich Verantwortung übernehmen und handeln. Die Sicherheit von Radfahrern, Fußgängern und Anwohnern darf nicht länger dem Durchgangsverkehr geopfert werden. Neben baulichen Maßnahmen sind regelmäßige Geschwindigkeitskontrollen in den Sommermonaten essenziell. Verkehrssicherheit darf keine Option sein – sie ist eine Verpflichtung gegenüber \textbf{allen}, die sich auf unseren Straßen bewegen.
	
	
