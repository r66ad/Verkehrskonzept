\section{Mobilität und Verkehr}
\subsection{Förderung von Fuß- und Radverkehr}

In diesem Abschnitt werden Maßnahmen zur Verbesserung der Infrastruktur für Fußgänger und Radfahrer vorgestellt. Besondere Aufmerksamkeit wird dabei den oftmals missachteten Sicherheitsabständen von 1,5 m innerorts und 2 m außerhalb gewidmet. Engpässe und Gefahrenstellen, an denen die Einhaltung dieser Abstände derzeit unmöglich ist, werden identifiziert und Lösungsansätze vorgeschlagen.

Schlüsselmaßnahmen:
\begin{itemize}
\item Einrichtung von Fahrradstraßen auf bestehenden Kommunikationswegen
\item Installation von Fahrradladesäulen
\item Nutzung von Straßenflächen für Parkplätze
\end{itemize}

Die Einrichtung von Fahrradstraßen, im Gegensatz zu Fahrradwegen,  würde erlauben, den vorhandenen Raum effizienter zu nutzen, und somit eine integrative und nachhaltige Mobilitätslösung darstellen. Bedauerlicherweise fand der Vorschlag zur Einrichtung einer Fahrradstraße in Kövenig und Reil keine Zustimmung seitens der politischen Entscheidungsträger. Das ist eine enttäuschende Situation, aber keineswegs unüberwindbar. In vielen Gemeinden gibt es anfänglich Widerstand gegen neue Verkehrskonzepte, insbesondere wenn diese eine Umstellung der bestehenden Verkehrsgewohnheiten erfordern. Dennoch gibt es Wege, die öffentliche und politische Meinung in diese Richtung zu lenken.

Hier ein paar Vorschläge zur weiteren Vorgehensweise:
\begin{description}

\item[Aufzeigen der Vorteile]
Die Vorteile einer Fahrradstraße sollten klar und unmissverständlich dargestellt werden. Dazu gehören erhöhte Verkehrssicherheit, Reduzierung der CO2-Emissionen und Steigerung der Lebensqualität. 

\item[Bürgerbeteiligung]
Durch die Einbindung der Bürgerinnen und Bürger in den Entscheidungsprozess kann das Bewusstsein für die Dringlichkeit der Maßnahmen geschärft werden. Öffentliche Versammlungen, Bürgerbefragungen oder Workshops können dazu beitragen.

\item[Datenbasierte Argumentation]
Verkehrszählungen und wissenschaftliche Studien können die Notwendigkeit einer Fahrradstraße untermauern. Solche Daten können die Politiker unter Umständen umstimmen.

\item[Pilotprojekt]
Der Vorschlag einer temporären Fahrradstraße als Pilotprojekt wurde bereits vorgelegt. Dieses Element sollte wieder in den Fokus gerückt und möglicherweise medial hervorgehoben werden, um sowohl die Durchführbarkeit als auch die vorteilhaften Auswirkungen zu unterstreichen.

\item[Alternativvorschläge]
Sollte keine der vorgeschlagenen Lösungen greifen, könnte als Kompromiss eine \textbf{"Tempo-30-Zone zwischen Reil, Kövenig und der Schleuse Enkirch"} in Erwägung gezogen werden. Außerdem wurde bereits auch eine "probeweise eingerichtete Fahrradstraße" diskutiert und würde damit eine sinnvolle Option darstellen. Es ist wichtig zu betonen, dass gerade auf dieser Strecke im Sommer viele Familien mit Kindern sowie Rentnerinnen und Rentner unterwegs sind. Diese Gruppen sind oft nicht so versiert im sicheren Radfahren. Eine Geschwindigkeitsreduzierung würde daher nicht nur die allgemeine Verkehrssicherheit erhöhen, sondern auch dazu beitragen, riskantes Fahren mit Geschwindigkeiten von über 70 km/h im Ortsbereich zu unterbinden.

\item[Unterstützung von Experten]
Die Einbindung von Verkehrsplanern, Umweltaktivisten und anderen Experten kann die Glaubwürdigkeit und Überzeugungskraft der vorgeschlagenen Maßnahmen erhöhen.
\end{description}

Die Installation von \textbf{Fahrradladesäulen} würde nicht nur die Elektromobilität im Ort fördern, sondern auch ein deutliches Signal für den dringend benötigten infrastrukturellen Wandel setzen. Während in Kröv bereits mehrere solcher Ladesäulen existieren, fehlt eine solche Infrastruktur bislang in Kövenig. Ein geeigneter Standort für die Errichtung der Fahrradladesäulen könnte das Gemeindehaus sein. Dabei könnten auf dem zugehörigen Parkplatz auch mehrere 11 /22kWh Type 2 Ladesäulen integriert werden.

\textbf{Die zweckmäßige Nutzung vorhandener Straßenflächen für Parkplätze} könnte als ein Schlüssel zur Lösung der komplexen Verkehrs- und Parkprobleme in den Moselorten dienen. Die Idee besteht darin, bereits asphaltierte Flächen als Parkraum zu nutzen, statt zusätzliche Grünflächen für diesen Zweck zu versiegeln. Diese Maßnahme steht in Einklang mit Prinzipien der Nachhaltigkeit und würde zum Erhalt der natürlichen Umgebung beitragen.

Darüber hinaus hätte die Nutzung von Straßenflächen für Parkplätze einen verkehrsberuhigenden Effekt. Durch die Einrichtung von Parkbuchten oder quer zum Gehweg angelegten Parkplätzen würde die Straßenbreite variieren, was die Durchfahrtsgeschwindigkeit der Fahrzeuge automatisch reduziert. Dies könnte besonders in Orten wie Kövenig, wo überhöhte Geschwindigkeiten ein ernstes Problem darstellen, zur Verkehrssicherheit beitragen. 

Es ist jedoch wichtig, diese Maßnahme im Kontext einer gesamtheitlichen Verkehrs- und Raumplanung zu betrachten (Stichwort Feuerwehreinfahrt), die auch den Fuß- und Radverkehr sowie den öffentlichen Nahverkehr berücksichtigt. Nur so kann ein ausgewogenes und nachhaltiges Mobilitätskonzept geschaffen werden, das den verschiedenen Bedürfnissen der Bevölkerung gerecht wird.
Diese Methode des Shared Space kann nicht nur die Geschwindigkeit des Straßenverkehrs reduzieren, sondern auch den öffentlichen Raum effizienter nutzen. Dadurch entsteht eine Win-Win-Situation, die die Lebensqualität in Kövenig und den umliegenden Orten spürbar erhöht.
\subsection{Gebührenpflichtiges Parken}
In Anbetracht der begrenzten Parkmöglichkeiten und der notwendigen Eindämmung des Autoverkehrs empfehlen wir die Einführung eines gebührenpflichtigen Parkens in Kövenig und den umliegenden Moselorten. Durch die Nutzung digitaler Plattformen wie 'Parkster' können die Parkvorgänge bequem über das Smartphone gesteuert werden, was die Administration für die Gemeinde vereinfacht und gleichzeitig eine gerechtere Parkraumnutzung fördert. Die daraus generierten Einnahmen könnten in nachhaltige Verkehrsprojekte und die Aufwertung des öffentlichen Raums reinvestiert werden. Zudem würde die Gebührenpflicht dazu führen, dass Anwohner ihre Garagen tatsächlich als Parkflächen nutzen und gegebenenfalls auf den Besitz eines Zweitwagens verzichten. Durch die Ausweisung spezifischer, gebührenpflichtiger Parkflächen würden zudem die Wildparkerei und die damit verbundenen Sicherheitsrisiken reduziert. Die Parkgebühren könnten dabei staffelweise und abhängig von der Parkdauer gestaltet sein, sodass sowohl kurzfristige Erledigungen als auch längere Aufenthalte berücksichtigt werden.

\subsection{Einführung von Tempo-30-Zonen}
Die Einführung von Tempo-30-Zonen stellt eine vielversprechende Maßnahme zur Verbesserung der Verkehrssicherheit und Lebensqualität in den Moselorten dar. Ein Beispiel: Die Erfahrung eines Besuchs in Traben Trarbach kann deutlich getrübt werden, wenn man ständig Autos ausweichen muss. Um die Lebensqualität und Sicherheit für alle Verkehrsteilnehmer zu erhöhen, sollte in sensiblen Bereichen wie auf der Brücke, der Bahnstraße und dem Marktplatz eine Geschwindigkeitsbegrenzung auf Schrittgeschwindigkeit eingeführt werden. Ebenso sollte in Kövenig zur Förderung der allgemeinen Verkehrssicherheit und des Umweltschutzes der gesamte Ort als Tempo-30-Zone ausgewiesen werden.


\textbf{Argumente für Tempo-30-Zonen:}
\begin{description}
\item[Sicherheit:] In Anbetracht der engen Gassen und des Fehlens von Bürgersteigen wäre eine Reduzierung der Geschwindigkeitslimits ein wesentlicher Schritt zur Verbesserung der Verkehrssicherheit für alle. Vor allem in der touristischen Saison könnte dies erheblich zur Sicherheit von Fußgängern und Radfahrern beitragen, darunter auch Familien mit Kindern und ältere Menschen, die nicht immer sicher das Fahrrad steuern können.

\item[Tourismus:] Eine reduzierte Geschwindigkeitszone könnte den Ort für Touristen attraktiver machen, indem die Sicherheit und Lebensqualität erhöht werden. Ein entspannteres Verkehrsumfeld könnte sich positiv auf die lokale Wirtschaft und den Tourismus auswirken.

\item[Umweltschutz:] Langsameres Fahren führt zu einer Reduzierung des Lärm- und Schadstoffausstoßes, was im Einklang mit den allgemeinen Zielen des Klima- und Verkehrskonzepts steht.

\item[Förderung des Nicht-Motorisierten Verkehrs:] Eine Geschwindigkeitsbegrenzung könnte die Attraktivität des Radfahrens und Zu-Fuß-Gehens erhöhen und somit zu einer Verkehrsverlagerung weg vom Auto beitragen.
\end{description}
Durch die Anpassung der Geschwindigkeitslimits an die spezifischen Bedingungen in Kövenig könnten sowohl die Sicherheit als auch die Lebensqualität erheblich gesteigert werden. Eine Tempo-30-Zone würde somit den Zielen einer nachhaltigen, umweltschonenden und sicheren Verkehrspolitik entsprechen.

\subsection{Verkehrsberuhigte Bereiche}
In Anliegerstraßen, wie bspw. Kirchstraße, sollten verkehrsberuhigte Bereiche geschaffen werden, um die Lebensqualität der Anwohner zu erhöhen und gleichzeitig die Verkehrssicherheit zu verbessern. Die Einführung solcher Zonen würde nicht nur den Lärmpegel und die Luftverschmutzung reduzieren, sondern auch das Risiko von Verkehrsunfällen minimieren. Dies ist besonders relevant in Straßen, die von Kindern, älteren Menschen oder Haustieren frequentiert werden. Darüber hinaus könnte dies einen positiven Nebeneffekt für die lokale Biodiversität haben, da verkehrsberuhigte Bereiche weniger anfällig für Straßenkill sind und damit einen sichereren Lebensraum für städtische Fauna bieten. Eine Umgestaltung der Straßenräume mit Grünelementen kann zusätzlich die Aufenthaltsqualität steigern und den städtischen Raum nachhaltig prägen.
Die Durchfahrt in verkehrsberuhigte Bereiche und Anliegerstraßen kann durch verschiedene Maßnahmen eingeschränkt werden, um den unerwünschten Verkehr durch Touristen oder Durchreisende zu vermeiden. Hier sind einige Optionen:
\begin{itemize}
\item\textbf{Poller und Barrieren:} Durch versenkbare Poller oder manuelle Barrieren kann der Zugang kontrolliert werden. Diese können so programmiert werden, dass sie zu bestimmten Zeiten oder auf Anforderung der Anwohner hin abgesenkt oder angehoben werden.

\item\textbf{Kennzeichnung und Beschilderung:} Klare Schilder, die auf die verkehrsberuhigte Zone und die Zufahrtsbeschränkungen hinweisen, können schon im Vorfeld abschreckend wirken. 

\item\textbf{Temporäre Straßensperren:} Zu bestimmten Tageszeiten oder an Wochenenden könnten Straßen temporär für den Durchgangsverkehr gesperrt werden, um nur den Anwohnern die Durchfahrt zu ermöglichen.

\item\textbf{Blumenkübel oder andere Gestaltungselemente:} Diese können nicht nur die ästhetische Qualität der Straße erhöhen, sondern auch als physische Barrieren dienen, die die Durchfahrt erschweren.
\end{itemize}