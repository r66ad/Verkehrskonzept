	\section{Analyse der Ausgangssituation}
\subsection{Aktuelle Verkehrssituation in Kövenig}
\subsection{Klima- und Umweltauswirkungen des Verkehrs}
\subsection{Herausforderungen und Chancen für nachhaltige Mobilität}

\section{Mobilität und Verkehr}
\subsection{Förderung von Fuß- und Radverkehr}
\subsection{Ausbau des öffentlichen Nahverkehrs}
\subsection{Elektromobilität und alternative Antriebe}
\subsection{Verkehrsberuhigung und Verkehrslenkung}

\section{Fahrradinfrastruktur}
\subsection{Ausbau von Fahrradwegen und Fahrradstraßen}
\subsection{Fahrradparkplätze und -stationen}
\subsection{Sicherheitsmaßnahmen für Radfahrer}

\section{Öffentlicher Nahverkehr}
\subsection{Verbesserung der Busverbindungen}
\subsection{Integration von umweltfreundlichen Mobilitätslösungen}

\section{Elektromobilität und alternative Antriebe}
\subsection{Ladeinfrastruktur für Elektrofahrzeuge}
\subsection{Förderung von Elektromobilität in der Gemeinde}

\section{Verkehrsberuhigung und Verkehrslenkung}
\subsection{Einführung von Tempo-30-Zonen}
\subsection{Verkehrsberuhigende Maßnahmen in sensiblen Bereichen}

\section{Umweltaspekte und Klimaschutz}
\subsection{Emissionsreduktion im Verkehr}
\subsection{Förderung von umweltfreundlichen Verkehrsalternativen}
\subsection{Beiträge zur CO2-Reduktion in Kövenig}

\section{Partizipation und Kommunikation}
\subsection{Einbeziehung der Bürgerinnen und Bürger}
\subsection{Öffentlichkeitsarbeit für das Klima- und Verkehrskonzept}

\section{Umsetzung und Zeitplan}
\subsection{Priorisierung der Maßnahmen}
\subsection{Schritte zur Umsetzung des Konzepts}
\subsection{Zeitschätzung für die Umsetzung einzelner Maßnahmen}

\section{Monitoring und Evaluation}
\subsection{Fortlaufende Überwachung der Maßnahmen}
\subsection{Anpassung des Konzepts bei Bedarf}
\subsection{Messbare Ziele für Klima- und Verkehrsentwicklung}

\section{Schlussbetrachtung}
\subsection{Zusammenfassung der wichtigsten Punkte}
\subsection{Ausblick auf die Zukunft von Kövenig als nachhaltiger Moselort}
