	\section{Einleitung}
	\subsection{Hintergrund und Zielsetzung}
	Die vorliegende Ausarbeitung präsentiert ein integriertes Klima- und Verkehrskonzept für die malerischen Moselorte, insbesondere Kövenig. In einer Ära, in der der Klimawandel unsere Lebensgrundlagen und natürlichen Ressourcen bedroht, ist es unerlässlich, zukunftsfähige und nachhaltige Lösungen zu finden. Wie der deutsche Astronaut Alexander Gerst aus dem All bemerkte: 'Man sieht Dinge, die einen überraschen: Gletscher die kleiner werden, Seen, die austrocknen.' Claudia Kemfert, eine führende Expertin für Energieökonomie, betont, dass uns diese Worte und die Satellitenbilder unseres Planeten aufrütteln müssen.\cite{kemfert2023schockwellen}
	
	Unser Konzept baut auf dieser Dringlichkeit auf und verfolgt einen dreigliedrigen Ansatz: Erstens die Etablierung einer nachhaltigen Verkehrsinfrastruktur, zweitens die partizipative Einbindung der Gemeinschaft in Entscheidungsprozesse und drittens die konsequente Ausrichtung an den Prinzipien der Nachhaltigkeit. Unsere Vision ist nicht weniger als die Transformation Kövenigs und der umliegenden Moselorte in Vorzeigeregionen für nachhaltige ländliche Entwicklung. Wir schlagen einen breiten Mix an Maßnahmen vor, die von Geschwindigkeitsreduktionen bis hin zur Elektromobilität reichen, um so eine sichere, ökologisch verträgliche und sozial gerechte Zukunft für alle Bürgerinnen und Bürger zu gewährleisten.
	\subsection{Bedeutung von Klimaschutz und nachhaltiger Mobilität in der Kommunalpolitik}
	Klimaschutz und nachhaltige Mobilität sind heute von essentieller Bedeutung in der Kommunalpolitik. Eine besorgniserregende Entwicklung in Kövenig ist die zunehmende sinnlose Flächenversiegelung. Dabei werden wertvolle Grünflächen für das Parken genutzt, ohne eine nachhaltige Verkehrs- oder Umweltstrategie zu berücksichtigen. Diese Versiegelung von Grünflächen hat nicht nur negative Auswirkungen auf das Dorfbild, sondern auch auf die Umwelt. Durch die Versiegelung werden natürliche Abfluss- und Filtersysteme gestört, was zu einer erhöhten Wasserbelastung, Bodenerosion und letztendlich zu einem Verlust an Biodiversität führt. Eine sinnvolle Planung und Gestaltung von Parkflächen unter Einbeziehung ökologischer Aspekte ist daher unerlässlich, um langfristig lebenswerte und nachhaltige Lebensräume zu erhalten. Siehe dazu \cite{web:umweltamt:nichtinvestiv}
\blockquote{Fläche: Straßenbauten und sonstige Verkehrsinfrastruktur führen zur Versiegelung von Flächen und somit zu einem dauerhaften Flächenverbrauch: Zum einen stehen versiegelte Flächen nicht mehr für Landwirtschaft, Freizeit und Erholung zur Verfügung. Zum anderen gehen wichtige Bodenfunktionen verloren. Der Boden ist ein bedeutender Speicher für Kohlenstoff, Nährstoffe und Wasser, er bindet Schadstoffe und reinigt das Trinkwasser. Versiegelung und Flächenverbrauch stellen also einen tiefgreifenden Einschnitt in das ökologische Gefüge dar.}
Die zurückliegende Ablehnung der Fahrradstraße zwischen Reil und Kövenig, sowie die Diskussionen um den nachhaltigen Tourismus in unserer Region haben verdeutlicht, dass eine Neuausrichtung unserer Verkehrs- und Mobilitätspolitik dringend erforderlich ist. Die Förderung des Radverkehrs, die Schaffung einer sicheren und umweltfreundlichen Infrastruktur sowie die Stärkung des öffentlichen Nahverkehrs sind Schritte, die nicht nur zur Verringerung der Umweltauswirkungen des Verkehrs beitragen, sondern auch das Image von Kövenig als nachhaltiger Ort stärken werden.


