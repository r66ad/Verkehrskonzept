	\section{Analyse der Ausgangssituation}
\subsection{Aktuelle Verkehrssituation in Kövenig}
  Die derzeitige Verkehrssituation in Kövenig ist geprägt von verschiedenen Herausforderungen und Problemstellungen, die sowohl die Sicherheit der Verkehrsteilnehmer als auch die Umweltauswirkungen des Verkehrs betreffen.
\begin{description}

  \item[Regelmäßige Überschreitungen der Geschwindigkeit]
  Bedauerlicherweise sind in unserem Ort regelmäßige Geschwindigkeitsüberschreitungen zu beobachten. Insbesondere in verkehrsberuhigten Zonen und innerörtlichen Straßen kommt es zu gefährlichen Situationen durch zu hohe Geschwindigkeiten. Die in der Tabelle~\ref{table:geschw} aufgeführten Geschwindigkeitsüberschreitungen verdeutlichen eine besorgniserregende Tendenz: Es fällt auf, dass viele Fahrzeuge mit Ortskennzeichen die innerörtliche Geschwindigkeitsbegrenzung von 30 km/h durch Kövenig deutlich überschreiten. Dies wirft Fragen bezüglich der Verkehrssicherheit und der Einhaltung von Tempolimits auf. Die Kennzeichen wurden zufällig während meiner Gassigänge mit dem Hund notiert, was die Relevanz und Authentizität dieser Aufzeichnungen unterstreicht. Dieser wiederholte Verstoß gegen die Geschwindigkeitsbegrenzung kann erhebliche Sicherheitsrisiken für die Bewohner und andere Verkehrsteilnehmer darstellen und \textbf{verdeutlicht die Notwendigkeit, Maßnahmen zur Geschwindigkeitsreduktion und -überwachung zu ergreifen}.



\begin{table}[htbp]
	\begin{tabular}{lllll}
		\toprule
		Datum  & Kennzeichen  & Geschwindigkeit \\
		\midrule
 11.02.2023 18:36 & Wil ap 2xx & 56kmh \\
 18.03.2023 14:15 & Wil as 9xx & 57kmh \\
 29.03.2023 08:56 & Wil xh xx & 66kmh \\
 29.03.2023 08:56 & Wil sf 2xx & 62kmh \\
 12.04.2023 09:08 & Wil mj 5xx & 55kmh \\
 09.05.2023 07:40 & Coc s 8xx & 65kmh \\
 12.05.2023 08:44 & Wil hy 2xx & 57kmh \\
 24.05.2023 17:36 & Zel a xx x & 69kmh \\
 26.05.2023 11:58 & Wil id 4xx & 61kmh \\
 31.05.2023 12:53 & Wil dr xx & 56kmh \\
 10.06.2023 08:25 & Coc i 3xx & 66kmh \\
 12.06.2023 08:32 & Bks id xx & 59kmh \\
 12.06.2023 20:25 & Bks s 18xx & 64kmh \\
 18.06.2023 10:05 & Bks fv xx & 59kmh \\
 20.06.2023 13:15 & Bks es 6xx & 59kmh \\
 23.06.2023 12:19 & Zel jm xx & 58kmh \\
 27.06.2023 07:01 & Zel pf xx & 67kmh \\
 28.06.2023 07:37 & Bks jg xx & 59kmh \\
 28.06.2023 07:34 & Myk gb 1xx & 59kmh \\
 29.06.2023 07:14 & Wil kj 8xx & 57kmh \\
 04.07.2023 18:27 & Bks co xx & 68kmh \\
 05.07.2023 17:06 & Wil pm 2xx & 56kmh \\
 06.07.2023 07:59 & Wil ml 8xx & 56kmh \\
 07.07.2023 07:44 & Bks jh xx & 54kmh \\
 10.07.2023 12:34 & Zel f 3xx & 56kmh \\
 11.07.2023 08:14 & Dau g 8xx & 50kmh \\
 20.07.2023 12:28 & Coc j 19xx & 60kmh \\
 17.08.2023 12:17 & Wil sb 2xx & 58kmh \\
 18.08.2023 12:39 & Bks cd x & 55kmh \\
 18.08.2023 12:47 & Zel mp xx & 50kmh \\
 22.08.2023 08:01 & Wil tj xx & 55kmh \\
 28.08.2023 12:35 & Wil cs 1xx & 55kmh \\
 29.08.2022 17:55 & Wil m 2xx & 57kmh \\
 06.09.2022 19:49 & Wil fd 1xx & 53kmh \\
 06.09.2022 09:02 & Zel da 1xx & 54kmh \\
 08.09.2022 17:55 & Zwl ar x & 55kmh \\
 08.09.2022 13:29 & Bks fd 1xx & 66kmh \\
 09.09.2022 20:23 & Wil cu xx & 51kmh \\
 09.09.2022 20:23 & Wil cu xx & 51kmh \\
 09.09.2022 20:23 & Wil aw 2xx & 54kmh \\
 18.09.2022 14:48 & Wil aw 2xx & 54kmh \\
 29.09.2022 08:30 & Wil ej xx & 57kmh \\
 14.10.2022 17:23 & Bks so xx & 65kmh \\

		\bottomrule
	\end{tabular}
	\caption{Zufällige Aufzeichnungen der Geschwindigkeitsüberschreitungen in der 30er Zone - Kövenig}
	
	\label{table:geschw}
	
\end{table}

\item[Sicherheitsmängel beim Überholen von Radfahrern] In der aktuellen Verkehrssituation in Kövenig ist ein besonderes Problem die Missachtung der vorgeschriebenen Mindestabstände beim Überholen von Radfahrern. Laut Verkehrsregeln sollten innerorts mindestens 1,5 Meter und außerorts mindestens 2 Meter Abstand beim Überholen eingehalten werden. Leider wird diese Regelung häufig ignoriert. Noch problematischer ist, dass die Enge der Straßen in vielen Fällen gar nicht ausreicht, um die vorgeschriebenen Sicherheitsabstände einzuhalten. Dies führt nicht nur zu einem erhöhten Unfallrisiko, sondern schafft auch ein Klima der Unsicherheit für Radfahrer. Insbesondere riskante Überholvorgänge vor nicht einsehbaren Kurven verschärfen die Gefahr und tragen zur Verkehrsunfallstatistik bei.

  \item[Steigende Zahl der Radfahrer]
  Die steigende Anzahl von Radfahrern in Kövenig ist erfreulich, jedoch führt die mangelnde Infrastruktur für Radfahrer zu Konfliktsituationen und Unsicherheiten im Straßenverkehr.

  \item[Fehlender Bürgersteig und hohe Geschwindigkeiten]
  Ein weiteres drängendes Problem ist der fehlende Bürgersteig in großen Teilen des Ortes, vor allem an Stellen, an denen regelmäßig mit über 70 km/h innerorts gefahren wird. Dies stellt sowohl für Fußgänger als auch für Radfahrer eine erhebliche Gefahr dar.

  \item[Anlieger-freie Straßen und Durchgangsverkehr]
  Straßen, die eigentlich als Anlieger-frei gelten sollten, werden bedauerlicherweise von anderen Verkehrsteilnehmern als Durchgangsroute genutzt. Dies führt zu zusätzlicher Verkehrsbelastung und Unannehmlichkeiten für die Anwohner.

  \item[Parkplatzsituation]
  ...und inkonsistente Prioritätensetzung in der Verkehrs- und Raumplanung.
  "Wenn ich mein nicht fahrbereites Sofa auf einem 12qm großen Platz im öffentlichen Raum abstellen würde, wäre das eine Ordnungswidrigkeit.\cite{diehl2022autokorrektur}
  Die Parkplatz- und Verkehrssituation in Kövenig ist nicht nur ein Spiegel der komplexen Herausforderungen urbaner Mobilität, sondern auch ein Indikator für die Unausgewogenheit in der Prioritätensetzung der Verkehrsplanung. Erstaunlich schnell wurde eine Nothaltebucht errichtet, die weitere wertvolle Grünflächen versiegelt hat. Dies steht im Widerspruch zu den dringenden Bedürfnissen der Gemeinschaft, wie der Reduzierung der Fahrgeschwindigkeit, dem Ausbau von Fahrradstraßen und der Schaffung strategisch platzierter Parkplätze zur Verkehrsentschleunigung. Während ad-hoc-Maßnahmen schnell umgesetzt werden, finden nachhaltigere und dringendere Initiativen wenig Beachtung. Dies stellt eine vertane Chance dar, sowohl im Hinblick auf die Verkehrssicherheit als auch im Kontext der ökologischen Nachhaltigkeit und der Verbesserung der Lebensqualität.
\end{description}
\subsection{Klima- und Umweltauswirkungen des Verkehrs}
  Der Verkehr hat erhebliche Auswirkungen auf Klima und Umwelt. Der Großteil der Treibhausgasemissionen in Deutschland stammt aus dem Verkehrssektor. Laut dem Umweltbundesamt verursacht der Straßenverkehr allein etwa 18 Prozent der CO2-Emissionen in Deutschland\cite{web:umweltamt:nichtinvestiv}. Diese Emissionen tragen maßgeblich zur Klimaerwärmung und den damit verbundenen Folgen wie Hitzewellen, Dürren und Meeresspiegelanstieg bei.

  Die Umweltauswirkungen des Verkehrs beschränken sich jedoch nicht nur auf CO2-Emissionen. Luftschadstoffe wie Stickoxide (NOx) und Feinstaub können zu Gesundheitsproblemen wie Atemwegserkrankungen und Herz-Kreislauf-Beschwerden führen. Lärmbelastung durch Verkehr verursacht ebenfalls negative Auswirkungen auf die Gesundheit und Lebensqualität der Bevölkerung.

  Um diese negativen Folgen zu reduzieren, ist eine umfassende Verkehrswende hin zu nachhaltigen und emissionsarmen Verkehrsmitteln dringend erforderlich. Dies erfordert eine enge Zusammenarbeit zwischen der Gemeinde, der Wirtschaft und den Bürgerinnen und Bürgern, um innovative Lösungen zu entwickeln, die sowohl die Mobilitätsbedürfnisse als auch den Umweltschutz berücksichtigen.

\subsection{Herausforderungen und Chancen für nachhaltige Mobilität}
  Die Umstellung auf nachhaltige Mobilität in Kövenig birgt sowohl Herausforderungen als auch Chancen. Eine der größten Herausforderungen besteht in der Änderung des Verhaltens der Verkehrsteilnehmer und der Akzeptanz neuer Mobilitätskonzepte. Gewohnheiten wie das Autofahren müssen überdacht werden, und es bedarf einer gezielten Sensibilisierung und Aufklärung der Bürgerinnen und Bürger.

  Gleichzeitig eröffnen sich jedoch auch zahlreiche Chancen. Die Förderung von Fahrrad- und Fußverkehr kann nicht nur die Umweltauswirkungen des Verkehrs reduzieren, sondern auch die Gesundheit der Menschen fördern und die lokale Wirtschaft durch den Tourismus stärken. Und während der Bahnverkehr in Kövenig bereits regelmäßig stattfindet, bietet die Elektrifizierung der Bahnstrecken eine Möglichkeit zur weiteren Verbesserung der ökologischen Nachhaltigkeit. Darüber hinaus könnten für die lokale Bevölkerung sowie für Touristen vergünstigte Fahrscheine für die Bahn (beispielsweise das 49EUR Ticket gültig auch für die Fähre) angeboten werden. Diese Maßnahmen würden nicht nur die Nutzung des öffentlichen Nahverkehrs attraktiver gestalten, sondern auch die Verkehrsbelastung auf den Straßen verringern, was zu einer ganzheitlich effizienteren Mobilitätslösung beiträgt. Neue Technologien wie Elektromobilität und innovative Sharing-Konzepte bieten die Möglichkeit, umweltfreundliche Optionen attraktiver zu gestalten.

Die Herausforderungen sollten als Anreiz für kreative Lösungen dienen, die sowohl die Mobilitätsbedürfnisse als auch die Umweltauswirkungen berücksichtigen. Mit einer klaren Vision und dem Willen zur Veränderung kann Kövenig eine Vorreiterrolle in Sachen nachhaltige Mobilität einnehmen und damit langfristig eine lebenswerte und zukunftsorientierte Gemeinde gestalten.